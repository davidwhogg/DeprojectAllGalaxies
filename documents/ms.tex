% This file is part of the DeprojectAllGalaxies project.
% Copyright 2015 the authors (see below)

\documentclass[12pt]{article}
\usepackage{url}
\input{vc}

\newcommand{\project}[1]{\textsl{#1}}
\newcommand{\illustris}{\project{Illustris}}

\begin{document}

\section*{Blind deprojection of diverse galaxies \\ given heterogeneous imaging}
\noindent
Dalya Baron \\
\textsl{Tel Aviv University}
\\[1ex]
David W. Hogg \\
\textsl{New York University} \\
\textsl{Simons Center for Data Analysis}
\\[1ex]
\texttt{draft 2015-10-02}

\paragraph{Abstract:}
Astronomical images of galaxies can be thought of as two-di\-men\-sion\-al
projections of three-dimensional objects, with each image a unique
projection of a suigeneris object.
If subsets of galaxies have strong morphological similarities in the
three-dimensional space, or if the three-dimensional galaxy population
can be approximated as being generated by a number of
three-dimensional archetypes, then it might be possible to cluster
images and deproject them statistically to find the three-dimensional
structures of the archetypes.
This problem relates to the problems of tomographic image
reconstruction and cryo-electron microscopy; in both of these the
standard approaches operate in Fourier space.
Here we take a real-space, probabilistic-inference approach to this
problem, defining a likelihood function for a collection of images,
parameterized by projection parameters and three-dimensional
galaxy-morphology parameters.
We demonstrate with artificial data generated from the
\illustris\ suite of simulations that we \emph{can} perform this blind
deprojection in practice, simultaneously classifying images and
deprojecting archetypes, both accurately.

\section{Introduction}

Tomography, theorems and methods.

Microscopy.

Rybicki note.  What other prior work is there?

\section{Assumptions and Methods}

Images are two-dimensional projections of three-dimensional galaxies,
convolved with a point-spread function and noisified with Gaussian
noise.

Projection directions are isotropic in the natural Euler-angle sense.

Galaxies are optically thin in the relevant sense.

There are $M$ galaxy archetypes.

The galaxies and the PSF can be represented well with mixtures of
Gaussians.

\section{Experiments}

Make data from \illustris\ sims?

Show that we can deproject, given images and projection parameters.

Show that we can classify images and determine projection parameters,
given galaxies and images.

Show that we can do it all.

Run on some real data.

\section{Discussion}

Re-visit the assumptions.

Compare to prior work.

All code and data used in this project are available
online\footnote{\giturl\\This version of the manuscript and results
  were generated with the git repository at hash
  \texttt{\githash~(\gitdate)}}.

\paragraph{Acknowledgements:}
This project was started at \project{AstroHackWeek 2015} at New York
University 2015 September 28 through October 02.
The authors are very grateful for helpful discussions with Leslie
Greengard (SCDA).

\end{document}
